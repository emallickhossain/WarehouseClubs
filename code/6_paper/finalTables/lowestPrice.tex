\begin{table}[!htbp] \centering
\caption{First-Best Savings by Household Income and Product}
\label{tab:excessSpending}
\begin{tabular}{lcccc}
\\[-1.8ex]\hline
\hline \\[-1.8ex]
            & \multicolumn{2}{c}{Non-Perishable} & \multicolumn{2}{c}{Perishable} \\
              \cline{2-3}                             \cline{4-5}
Income      & Toilet Paper  & Diapers & Milk  & Eggs \\
\hline
$<$25k      & 0.36          & 0.33    & 0.31  & 0.17 \\
25-50k    & 0.35          & 0.33    & 0.30  & 0.17 \\
50-100k   & 0.34          & 0.33    & 0.31  & 0.17 \\
$>$100k     & 0.33          & 0.31    & 0.33  & 0.18  \\
\\[-1.8ex]\hline
\hline \\[-1.8ex]
\end{tabular}
\begin{tablenotes}
Table reports average savings a household could achieve given its brand and store choice. Average savings for a family of four is reported above. For example, a household making $<$\$25k could save 36\% by purchasing at the lowest unit price available.
\end{tablenotes}
\begin{tablenotes}[Source]
Nielsen Consumer Panel (2006--2016) and Nielsen Retail Scanner (2006--2016)
\end{tablenotes}
\end{table}
